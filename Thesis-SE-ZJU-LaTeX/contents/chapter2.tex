%!TEX root = ../main.tex

\chapter{结合强化学习的图结构大语言模型总体方案设计\footnote{请不要在正式论文撰写中如本文般滥用脚注}}
% 原则上,每一章开头不可马上进入第一节
% 需要一段引文提要起到 TL;DR 的作用
本章围绕结合强化学习的图结构大语言模型总体方案展开。
首先,对面向图结构推理的建模任务进行形式化描述,解释方法流程框架并理清数据流脉络。
随后,本章从整体框架层面详细介绍了图结构编码、图–语言表征融合以及后训练与强化学习策略之间的协同关系,
勾勒模型训练与推理的完整流程。
在此基础上,给出本研究采用的评估指标体系与数据配置方案,
包括真实图数据与合成图推理数据的使用方式及划分原则,
为后续章节中具体方法的设计、实验设置与结果分析提供统一的任务背景与实验基础。


\section{问题描述}


本研究面向文本属性图(Text-Attributed Graphs, TAGs)上的图结构理解与推理任务,
旨在构建一个统一的图—语言模型框架,使模型在处理图数据时能够同时兼顾预测性能、
推理可解释性与跨任务泛化能力。

形式化地,给定一张文本属性图
\(
G = (V,E,A,T),
\)
其中 \(V=\{v_i\}_{i=1}^{N}\) 为节点集合,\(N = |V|\) 为节点数;
\(E \subseteq V \times V\) 为边集合,表示节点之间的关系(如引用关系、共购关系等);
\(A \in \{0,1\}^{N \times N}\) 为图的邻接矩阵;
\(T = \{t_i\}_{i=1}^{N}\) 为节点文本属性集合,\(t_i\) 表示与节点 \(v_i\) 相关的自然语言描述。
我们希望学习一个参数为 \(\theta\) 的图基础模型(Graph Foundation Models,GFMs)
\(
F_{\theta} : (G,q) \rightarrow (c,a),
\)
其中 \(q\) 为关于图的自然语言指令或问题,\(a\) 为模型给出的答案,
\(c=(c_1,\dots,c_L)\) 为可选的链式思维推理过程(Chain-of-Thought),
用于刻画模型在图上的多步推理轨迹。
目标是在不同类型的图任务 \(\mathcal{T}\) 上,使 \(F_{\theta}\) 在准确性、可解释性和泛化能力之间取得合理平衡。

针对上述总体目标,本文将研究任务进一步拆解为以下几个子任务:

\textbf{1)节点分类任务:} 给定图 \(G\) 及其中一部分带标签节点集合
\(
V_{\text{lab}} \subseteq V,
\)
每个标注节点 \(v_i \in V_{\text{lab}}\) 对应类别标签 \(y_i \in \mathcal{Y}\)
(\(\mathcal{Y}\) 为类别空间)。
模型需要利用图结构 \(A\)、节点文本属性 \(T\) 以及自然语言提示 \(q\),
学习条件分布
\(
p_{\theta}(y \mid G,v,q),
\)
并对未标注节点 \(v \in
V_{\text{unlab}} = V \setminus V_{\text{lab}},
\) 进行类别预测,
实现对图中节点语义与结构角色的统一建模。

\textbf{2)链路预测任务:}
在同一张图 \(G\) 上,给定若干节点对 \((v_i,v_j)\),其中部分为真实存在的边,
部分为负样本,模型需要估计边 \((v_i,v_j)\in E\) 出现的概率
\(
p_{\theta}\big((v_i,v_j)\in E \mid G,q\big) \in [0,1],
\)
据此完成缺失边的预测或潜在关系的发现。
相比传统仅依赖结构或嵌入的方法,本研究要求模型在推理过程中
同时利用图结构模式与节点文本含义,并可通过自然语言形式解释预测结果。

\textbf{3)图推理与问答任务:}
在更一般的设定下,给定图 \(G\) 及其文本属性,
以及一条关于结构或语义关系的自然语言问题 \(q\)
(如“节点 A 与节点 B 是否存在长度不超过 \(L\) 的路径?”
“该节点更接近哪一类研究方向?”等),
模型需要生成一段多步推理链 \(c\) 以及最终答案 \(a\):
\(
(G,q) \xrightarrow{F_{\theta}} (c,a).
\)
其中 \(c\) 应体现对图结构的逐步分析与决策依据,
\(a\) 需满足由图算法或规则可验证的正确性约束。
该子任务综合考察模型对图结构的深度理解、跨模态信息融合以及链式推理能力,
是评估 Graph-LLM 真实实用价值的核心场景。

综上,本课题的根本目标在于:
在统一的图—语言框架下,通过合理的图结构表示、
图与文本的表征对齐以及后续训练与强化学习策略,
使模型能够在上述多个子任务上表现出稳定、可解释且具备迁移能力的图推理行为。




\section{总体流程框架}
针对上述科学问题, 本文构建了一个三阶段渐进式训练框架,
旨在系统性解决图结构大语言模型在结构理解与推理泛化方面的核心挑战。 
其流程框架如图 2.1所示, 主要包括三个阶段: 无损图结构解析与表征预训练阶段、
多层次图—语言语义对齐阶段、面向图结构推理的大模型后训练阶段,
建立了从图结构表征到可验证推理链生成的完整技术路径。

\textbf{1)无损图结构解析与表征预训练阶段}

在第一阶段,针对文本属性图中局部子图结构,本文借鉴基于欧拉路径的图序列化思想,
构造对节点及其邻域的半欧拉路径表示,将原本离散的拓扑结构转换为可逆的序列输入。
与传统基于邻域模板的线性化方式相比,该方案尽可能保留节点与边的完整连接关系,
避免因采样和裁剪带来的结构信息丢失。
在此基础上,引入基于Transformer架构的图结构编码器,对欧拉序列进行预训练,使其学习到对度分布、
路径模式以及局部子结构等关键信息敏感的表征。
这一阶段的目标是在不依赖大规模人工标注的前提下,为后续图–语言融合提供结构保真、
可迁移的图表示基础。


\textbf{2)多层次图—语言语义对齐阶段}

第二阶段致力于图–语言多模态对齐。在保持预训练语言模型参数基本稳定的前提下,
将图编码器产生的结构表征映射到语言嵌入空间,并结合全局对比、
局部匹配以及生成式约束等多种训练信号,对图表示与文本表示进行联合优化。
通过这一过程,模型逐步建立起“图结构—自然语言描述—任务指令”三者之间的一致语义空间,
使得语言模型在接收图相关输入时能够“读懂”图特征,而不仅仅视其为无结构的附加向量。

\textbf{3)面向图结构推理的大模型后训练阶段}

第三阶段面向强化学习驱动的后训练。
在多种图推理与预测任务上,将图大语言模型视作策略生成器,
令其给出包含中间分析步骤的推理链及最终答案,
再借助图算法或规则系统对生成结果进行自动验证,
构造可度量的奖励信号。在此基础上,通过策略优化方法对模型进行迭代更新,
使其在满足结构约束的前提下,倾向于生成更准确、
更连贯且更具可解释性的推理过程,同时提升在未见图结构与新任务上的泛化能力。


\subsection{发行版和文本编辑器}
如果你能通过README编译出此份文档,
说明您的机器上已经安装好了一个\TeX 排版系统的发行版(比如MacTeX 201x)。
发行版融合了非常多的工具(包括命令行工具和窗口程序)和宏包,
您在本项目README中看到的\texttt{latexmk}就是其中包含的一个命令行工具。

% (如果您看不懂以下文字,可以忽略)
% 和 Linux 发行版一样,MacTeX 也支持软件包管理,
% 在 OS X 下,可以使用 MacTeX 自带的图形化包管理工具 TeX Live Utility,
% 也可以使用命令行工具\texttt{tlmgr}操作。

论文的\LaTeX 源码自然是纯文本,
您可以使用平时最习惯的文本编辑器编辑论文源码。
比如使用简单易用的 Atom 编辑器
\footnote{请不要陷入无意义的编辑器党争,适合自己的才是坠吼的。} % +1s 
。如果您是 windows 用户,请放弃使用默认的记事本程序,
此处不展开解释。

“纯文本”是使用\LaTeX 排版的第二个优势,
纯文本文件稳定易读,可以很方便地通过各种版本控制工具进行管理,
但请注意保护自己的论文源码,避免意外同步到线上公开仓库。

本模板使用\LaTeX 格式编写论文源码。
\label{dirtree}
接下来介绍一些\LaTeX 必要的语法和基本知识,
以便您在遇到问题时能尽量准确地描述,
以寻求社区或个人协助。
现在您可以分屏,分别阅读本文及其源码。
(本章源码在\texttt{contents/whyla.tex})

% TikZ 大法 先跳过阅读

\begin{figure}[htbp]
    \centering
    \begin{tikzpicture}[dirtree]
      \node {论文根目录}
        child { node {contents/}
            child{ node {abstract\_chinese.tex}}
            child{ node {abstract\_english.tex}}
% 如果您在这里忘记逃逸下划线,靠编译错误信息根本无法发现
% 如果您不知道我在这里说什么,请继续往后看
            child{ node {elem.tex}}
            child{ node {intro.tex}}
            child{ node {whyla.tex}}
            child{ node {rule.tex}}
            child{ node {end.tex}}
            child{ node {thanks.tex}}
        }
        child { node {main.tex}}
        child { node {figures/}}
        child { node {references/}}
        child { node {gbt7714-2005.bst}}
        child { node {zjuthesis.cls}};
    \end{tikzpicture}
    \caption{论文源码树}
\end{figure}

在文本编辑器中另开一个标签打开\texttt{main.tex},
从\texttt{\textbackslash documentclass\{...\}}
到\\\texttt{\textbackslash begin\{document\}}
之间的部分称为导言区,
此部分通常用于全文样式设定。
为最好的分离样式和内容,
本模板将样式文件独立放置于\texttt{zjuthesis.cls}
(请优先选用只读模式打开查看)。
在导言区之后,即可开始文档内容的撰写,
鉴于论文篇幅较大,本模板建议按章节分别编写。

\section{探索学习}
如本节标题,本文不会系统介绍\LaTeX 的各项语法标记。
若您觉得确有必要系统学习\LaTeX 语法知识,
请使用搜索引擎搜索:一份不太简短的 LaTeX 2ε 介绍。

通过调查本章源码,
相信你已经懂得如何开始编写
一章(\texttt{chapter})、
一节(\texttt{section})
和一小节(\texttt{subsection})。
% 如果需要编写更小的文档结构,
% 可以使用(\texttt{subsub-section})
% \footnote{此处标记实为\texttt{subsubsection}。
% 在连续排版等宽字体时,
% 断行算法容易失控造成溢出版心,
% 如果您认为应极力避免任何溢出版心的排版行为时,
% 请仅在文档内容稳定后再对此细节做调整。}。
和学院论文模板一样,本模板也不建议使用第四级标题x.x.x.x。
考虑到在论文第二章介绍相关技术时,
部分同学列举一些背景概念时可能需要使用更细粒度的结构,
本模板建议使用子段落(\texttt{subparagraph})来实现这个语义,如下。

% htbp 什么的现在不要管
\begin{figure}[htbp]
    \centering  % 学位论文规定图表皆水平居中于版心 在 zjuthesis.cls 搜「版心设置」
    \includegraphics[width = .4\linewidth]{plus-1.jpeg} % 设定图片宽度相对于版心宽度,图片文件资源名
    \caption{华莱士在其著作《马来群岛》中绘制的飞蛙速写} % 图的题注
    \label{fig:plus-1} % 与 autoref 关联,设定交叉引用和显示「图x.x」
\end{figure}


% 梦里不觉 ☐ 已深, ☐ ☐ 岂是为他人。
\subparagraph{华莱士飞蛙} % +1s
华莱士飞蛙 (Rhacophorus nigropalmatus) ,
得名于它的发现者——生物学家阿尔弗雷德 · 华莱士 (Alfred R. Wallace)。
华莱士飞蛙生活在马来半岛的森林里,
它的体型很大,体长有 8 到 10 厘米,
除了交配和产卵,它们几乎从不下树……
如\autoref{fig:plus-1} 所示。

您是否注意到了,
源码中单个换行符并不会被编译成文档中的换行,
而两个或者超过两个换行符将开启一个新段落。
(类似HTML中新建了一个\texttt{<p>}标签)。
如果需要在文章中随意插入一个换行(类似于\texttt{<br>}标签),
则应该在源码文件中编写$\backslash\backslash$实现。
此标记一般仅在排版表格时使用,
或者活用于后期微调工作。

由于\LaTeX 的命令会使用几个固定的字符,
同其他编程语言处理字符串时一样,
当输出此类字符时需要使用逃逸策略,
如果您的论文编译错误,请\textbf{优先检查是否忘记逃逸此类字符}
\footnote{鄙人在写这份文档的时候都还是会忘记记给下划线逃逸(\LaTeX 命令使用下划线表示下标)}。
逃逸规则见\autoref{tab:escape} 所示。


% 和图一样,这里的htbp先不要管,先照抄
\begin{table}[htbp]
    \centering  % 依照规定 表格必须居中版心放置
    \caption{\LaTeX 命令专用字符逃逸规则} % 表格题注,zjuthesis.cls 将其设置在表格之上
    \label{tab:escape} % 交叉引用
    \begin{tabu}{lllllllll} % 9 列均左对齐
        \toprule % 头线
        输出 & \#  & \$  & \&  & \_  & \{   & \}  & \~{}  & \`{} \\ % 注意到了这个换行符吗
        \midrule % 中线
        源码 &  % 感受一下这里的「逃逸」
        \texttt{$\backslash$\#} &
        \texttt{$\backslash$\$} &
        \texttt{$\backslash$\&} &
        \texttt{$\backslash$\_} &
        \texttt{$\backslash$\{} &
        \texttt{$\backslash$\}} &
        \texttt{$\backslash$\~{}} &
        \texttt{$\backslash$\`{}}\\
        \bottomrule % 底线
    \end{tabu}
\end{table}

至于反斜杠本身的逃逸方法,请见上一段文字或表格的源码。

默认情况下,
多个空格符、水平制表符或单个换行符,
都仅会被编译成文档中的一个空格,
然而,和西文不同,中文的词与词之间是没有空格的,
所以本模板在XeCJK环境的初始化配置上取消了此类规则。
利用这样的改动,
您在编写论文时可按照您的行文思路随意换行,
不必担心产生多余的空格。
随着您的行文篇幅渐巨,
您将慢慢体会到这种编辑方式的优势——
\textbf{内容}与\textbf{样式}的分离。

输出英文半角双引号,需要在源码编写两个反引号(``grave'')。
而中文双引号则直接在源码中编写一对全角双引号即可。
鉴于目前大陆地区的官方标准明确规定使用此种方法标记引号,
请不要在论文中使用台湾地区和日本等地使用的直角引号
\footnote{网页中文排版近年来有惯用直角引号的趋势,此处不讨论}。

% 论文除了第一章绪论和终章的总结与展望, 最后都需要撰写一节“本章小结”。
% 活用注释作为 ToDo 提醒是一个好习惯
\section{本章小结}
本章简单描述了\LaTeX 排版的基本概念和本模板的源码结构,
通过同步实例介绍了最基本语义单元的编写。
现在,您可以尝试开始编写自己的论文,
当遇到无法排版的元素或无法解决的问题时,
欢迎继续阅读下一章。
