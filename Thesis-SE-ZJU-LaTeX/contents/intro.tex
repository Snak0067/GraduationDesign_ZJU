% !TEX root = ../main.tex

% 第一章一般名为绪论/引言,不可省略

\chapter{绪论}

\section{研究背景}

在当前人工智能技术广泛赋能的时代,结构化图数据在多个关键领域中扮演着基础性角色。
社交网络中用户间的关注、互动构成异质社交图,用于社区检测与推荐系统 \cite{He2020LightGCNSA};
生物医药领域中分子结构与蛋白质交互天然以图形式建模,广泛用于药物筛选与分子生成\cite{Hu2020OpenGB};
交通调度与城市计算将路网与交通流抽象为图模型,用于路径优化与实时控制\cite{Zheng2019GMANAG};
金融风控利用交易网络进行欺诈检测和信用建模\cite{eFraudCom};
电子电路与EDA设计中电路拓扑本身即为图结构\cite{DeepGate}。
随着多领域数据规模与复杂性不断增长,图结构在表示复杂实体间多关系、高阶依赖方面展现出不可替代的表达能力。
因此,如何从图结构中高效抽取语义、理解拓扑规律,并迁移到各类图计算任务中,成为当前人工智能研究中的重要课题之一。

图结构数据的计算方法大体经历了三个发展阶段。首先是以传统图论和谱分析为代表的符号方法阶段。该阶段主要依托图的数学结构开展研究,
例如基于模块度和随机游走的社区划分方法、谱聚类与图切分、节点重要性排序、标签传播以及图同构与模式匹配等。这类方法具有较强的可解释性,但在处理大规模、动态或异质图时往往受限于计算复杂度和人工特征设计。
其次是以统计学习为核心的低维嵌入阶段。研究者通过将高维图结构映射为连续向量空间来表达节点、边或子图之间的关系,常采用随机游走与矩阵分解等方式获得节点的低维表示。这类方法能够较好地捕捉局部结构信息,便于在分类、聚类、链路预测和推荐等任务中使用,但仍然依赖人工设定的采样策略,难以充分反映图中复杂的高阶关系。
第三阶段是基于神经网络的图表示学习方法,即图神经网络的兴起。该方法以消息传递和邻域聚合为核心思想,通过迭代更新节点特征来学习图的结构与属性信息,从而实现端到端的任务建模。近年来,针对异质图、时序图以及大规模图的改进方法不断涌现,同时也结合自监督学习与对比学习来缓解标注数据不足的问题。
然而,这些传统图计算方法在面对复杂开放场景时仍存在诸多局限。一是表达能力有限,模型往往只能关注局部邻域,难以捕捉长距离依赖和全局结构信息;二是可扩展性不足,大规模图数据的计算开销大、采样偏差明显;三是语义层面存在割裂,结构特征与节点的文本或多模态信息难以有效融合;四是数据标注稀缺与分布不均,使得模型在冷启动和长尾节点上的表现受限。

\section{国内外研究现状}

对于国内硕士学位论文来说,
一般较少研究完全无前人探索的领域,
所以有必要交待前人在此做出的努力和尝试。
同样,请提供数据和引用保证严谨。

为避免引起评阅老师判定有凑篇幅之嫌,
请有针对的描述前人研究的不足之处,
做到``有破有立''。

\section{论文研究内容}

此部分必须详细描述,
必要时可划设小节。
国外学位论文的Introduction章基本仅阐述此内容。
为研究开展的相关工作和实验,
此间遇到何难处及对应的解法。
对论文研究领域不甚了解的评阅老师,
希望从摘要和此小节尽可能的了解最多信息。


\section{论文组织结构}

简明扼要的介绍下各章主旨,版面控制半页内。
