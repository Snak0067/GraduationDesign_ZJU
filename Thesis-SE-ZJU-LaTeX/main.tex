% !Mode:: "TeX:UTF-8"
% !TEX builder = LATEXMK
% !TEX program = xelatex
\documentclass[master,oneside,AutoFakeBold]{zjuthesis} % 如果你的论文不满80页,还是单面印刷吧


%%%%%%%%%%%%%%%%%%%%%%%%%%%%%% 开始填写前置部分使用的变量
%%%%%%%%%%%%%%%%%%%%%%%%%%%%%% 样式设定在 zjuthesis.cls 下, 人类可读,爱请查阅

% 这里写这么鬼畜是为了测试多几个字会不会造成溢出
\title{结合强化学习的图结构大语言模型关键技术研究} % 封面和题名页使用
\englishtitle{Key Techniques for Graph-Structured Large Language Models with Reinforcement Learning} % 封面和题名页使用
% 如果您的标题用字过多,请自行调节 zjuthesis.cls 里的 ZJUmakecover 里的各项距离。

%\author{伊藤春希}          % 申请人姓名 封面使用
\author{马晓峰}       

\classification{TP311.5}    % 封面头使用
\serialnumber{10335}        % 封面头使用
\secretlevel{无}            % 封面头使用
\studentnumber{22351147}    % 封面头使用

%\supervisor{御坂美琴}       % 导师 封面使用
\supervisor{耿卫东}  
%\spvtitle{电击使}           % 职称 封面使用
\spvtitle{教授}               % 职称 封面使用

%\cpsupervisor{桂和纱}       % 合作导师,如果没有合作导师,就在此文件第 4 行\documentclass选项栏中加上"nocpsupervisor"。
\cpsupervisor{师之名}
\cspvtitle{职称}             % 合作导师职称

% 从机械工程学院改来,保留设定变量命名
%\major{船舶工程}            % 专业学位类别栏 填 工程硕士
\major{电子信息}
%\research{白学}             % 专业学位领域栏 填 软件工程
\research{软件工程}
\institute{软件学院}         % 所在学位栏 填 软件学院

\submitdate{2026年3月30日}    % 论文提交日期 栏

% 题名页的评阅人及答辩席
% 归档时候填写
% 论文评阅人1 2 3 4 5
\reviewerA{隐名评阅} \enreviewerA{}
\reviewerB{隐名评阅} \enreviewerB{}
\reviewerC{隐名评阅} \enreviewerC{}
\reviewerD{} \enreviewerD{}
\reviewerE{} \enreviewerE{}

% 答辩委员会主席
\chairperson{倪超$\setminus$副教授$\setminus$浙江大学软件学院} \enchairperson{}

% 答辩委员 1 2 3 4 5
\commissionerA{曾丽敏$\setminus$副教授$\setminus$浙江大学软件学院} \encommissionerA{}
\commissionerB{张文$\setminus$副教授$\setminus$浙江大学软件学院} \encommissionerB{}
\commissionerC{房子荃$\setminus$助理研究员$\setminus$浙江大学软件学院} \encommissionerC{}
\commissionerD{周鑫崴$\setminus$高级工程师$\setminus$宁波市地方金融监管局} \encommissionerD{}
\commissionerE{} \encommissionerE{}

% 答辩日期
\defencedate{2026年3月1日} \eendefencedate{} % 因为endefencedate 命名被占用

% 论文前置部分变量填写完毕 开始全书排版
\begin{document}

% 封面、中文题名页、英文题名页、独创声明和版权使用书 无页码
\maketitle

% 摘要部分
\abstractmatter
% 定义中文摘要和关键字
\begin{cabstract}
    目前,图结构数据广泛存在于社交网络、引文网络与电商推荐等真实场景中,其节点通常附带丰富的文本属性,构成文本属性图(Text-Attributed Graphs, TAGs)。当前,图大语言模型(Graph-LLMs)已成为处理这类数据的核心技术,在学术引文分析、电商关联推荐、金融风险挖掘等领域展现出实用价值。
    
    图大语言模型核心是融合图神经网络(GNN)的结构建模能力与大语言模型(LLMs)的语义理解优势,通过图序列化转换、跨模态表征投影等方式,实现图拓扑关系与文本语义信息的联合处理。尽管大语言模型在自然语言处理任务中展现出强大能力,但其在图结构理解与推理方面仍面临显著挑战。现有 Graph-LLMs 普遍存在图结构编码信息损失、图文表征空间割裂、依赖高质量人工标注数据以及推理过程不可追溯等问题,严重制约了模型在复杂图任务中的泛化能力与实用性。
    因此,为了解决上述的问题和挑战,本文提出一种结合强化学习的图结构大语言模型关键技术框架,主要工作包括:

    1)图结构表征改进:针对传统图模型在序列化过程中存在结构信息损失的问题,本文引入基于欧拉路径的可逆图序列化方法,通过引入节点身份多重编码与可重采样机制,实现图结构的无损、可逆序列化,有效保留全局拓扑信息,为模型统一处理节点、边与图级任务奠定基础。实验证明,该方法在OGB-PCQM4Mv2与ogbl-ppa数据集上较X基线提升约X\%。

    2)图 - 文跨模态表征的多层次对齐改进。为弥合图嵌入与文本语义空间的模态鸿沟,设计多层次跨模态对齐策略,融合全局对比损失、生成式交叉熵损失与细粒度匹配损失,分别从全局语义相似性、局部 token 级对应性、语义生成完整性维度,弥合图结构拓扑特征与文本语义特征的异构鸿沟,将跨模态相似度提升至 0.65 以上,显著提升图结构与文本语义在嵌入空间的一致性。

    3)基准训练范式与强化学习优化。为增强模型在复杂图推理场景下的泛化能力,结合HPT统一训练框架,将监督微调与强化学习融合,在无需人工偏好标注的前提下,引导模型生成逻辑正确、可验证的推理链。实验结果表明,所提方法在Erdős图论推理基准及Cora、PubMed、OGB-Arxiv等真实图数据集上均取得显著性能提升,不仅增强了模型的零样本泛化能力,还实现了可追溯、高可信的图推理过程。

    最后,本文整合上述技术方案,设计并实现了一套统一图语言模型原型系统,整合集成了节点分类和链路预测等下游任务在内的相关技术方案,通过系统功能和性能测试,验证了所提出方案的有效性和实用性。
\end{cabstract}
    
\ckeywords{图大语言模型,文本属性图,跨模态对齐,强化学习}
\begin{eabstract}
    Currently, with the rapid development of generative artificial intelligence, generative digital humans have gradually widely applied across various industries. It can be utilized in many application scenarios such as short video generation, intelligent customer service systems. Realistic and good interactive generative digital humans can significantly improve the user experience in human-computer interaction.

    Generative digital human modeling relies on multi-dimensional features extracted from raw data, including expressions,hand and full-body gestures, as well as other dimensional features in temporal and spatial dimensions. In current work on generative digital human modeling, most studies have not fully leveraged the advantages of multi-dimensional features, instead using only single-dimensional features extracted from monocular video data sources or fusing only a few dimensional features. This leads to suboptimal digital human modeling results. Additionally, existing work lacks data processing methods that can directly extract multiple dimensional features from monocular videos. Specifically, this has led to current digital human modeling methods facing challenges in four aspects: controllability, generation quality, generation speed, and temporal consistency.
    
    To address the aforementioned problems and challenges, this thesis explores a digital human modeling solution based on multi-dimensional feature fusion. The main contributions include:
    
    1) Fusion and Enhancement of Global Body Features and Local Positional Features:
    Designed and implemented an adversarial generation network architecture based on multi-dimensional feature fusion balances generation quality and speed. For the task of multi-dimensional feature fusion involving global body features and local positional features, a loss function based on a large-scale human body model and a discriminator structure tailored to digital human tasks were developed. These improvements enhance both the local details and the overall appearance in digital human modeling for higher-quality digital human generation.
    
    2) Temporal Consistency Improvement through Fusion of Inter-frame Continuity and Normal Attributes:
    Designed and implemented a temporal feature fusion method based on 3D group convolution and a normal feature fusion method based on multi-task learning. These methods enhance the temporal consistency of generated digital human videos by addressing both frame consistency and cross-frame geometric consistency. Additionally, leveraging the benchmark dataset construction method presented in this thesis, a temporal closed-loop detection method was designed to further evaluate the temporal consistency of the generative digital human videos.
    
    3) Benchmark Dataset Construction Method and Benchmark Digital Human Dataset for Multi-dimensional Feature Fusion:
    Integrated multiple key technologies to achieve high-quality multi-dimensional synthetic digital human data representations extracted from monocular video data. These synthetic image representations accurately depict the digital human’s mouth, eyes, hands and body gestures, facilitating neural network modeling and significantly enhances the controllability of digital humans. Utilizing this benchmark dataset construction scheme, this thesis constructed a character dataset encompassing various scenarios such as speeches, sign language performances, and news broadcasts, covering both half-body and full-body characters. The dataset includes over 30 hours of high-quality digital human video data across 525 digital human identities and corresponding multi-dimensional synthetic images.
    
    Finally, the multi-dimensional feature fusion-based digital human modeling solution developed in this thesis has been adopted by companies such as Hangzhou Cultural Broadcasting Television Group and China Post Consumer Finance Company. For the customer service digital human scenario and the program production digital human scenario, corresponding digital human application solutions were developed, thereby validating the digital human modeling method presented in this thesis.
\end{eabstract}

\ekeywords{Generative Digital Human, Multi-dimensional Feature Fusion, Generative Artificial Intelligence}
    

% 目录和术语表
\frontmatter
%目录单倍行距设置
\begin{spacing}{1.2}
       \tableofcontents % 正文目录
       \listoffigures   % 图目录
       \listoftables    % 表目录
\end{spacing}

% 术语及缩略词表(需要则开)
%\include{contents/denotation}

% 正文排版开始 建议一章一文件 (好像无法嵌套 include) 
\mainmatter
% !TEX root = ../main.tex

% 第一章一般名为绪论/引言,不可省略

\chapter{绪论}

\section{课题背景与研究意义}

在当前人工智能技术广泛赋能的时代,结构化图数据\cite{图结构数据1,图结构数据2}在多个关键领域中扮演着基础性角色。
社交网络中用户间的关注、互动构成异质社交图,用于社交网络分析\cite{社交网络分析1,社交网络分析2}与推荐系统 \cite{He2020LightGCNSA,Wang2021LearningIB};
生物医药领域中分子结构与蛋白质交互天然以图形式建模,广泛用于生物网络分析\cite{Guo2022GraphbasedMR,Liu_Wang_Vu_Moretti_Bodenheimer_Meiler_Derr_2023}与分子生成\cite{Hu2020OpenGB};
交通调度与城市计算将路网与交通流抽象为图模型,用于路径优化与实时控制\cite{Zheng2019GMANAG};
金融风控利用交易网络进行欺诈检测和信用建模\cite{eFraudCom};
电子电路与EDA设计中电路拓扑本身即为图结构\cite{DeepGate}。
随着多领域数据规模与复杂性不断增长,图结构在表示复杂实体间多关系、高阶依赖方面展现出不可替代的表达能力。
因此,如何从图结构中高效抽取语义、理解拓扑规律,并迁移到各类图计算任务中,成为当前人工智能研究中的重要课题之一。

传统的图结构计算
主要依赖符号方法\cite{Blondel2008FastUO,随机游走,谱聚类,标签传播,图同构检测}与
统计学习方法\cite{Perozzi2014DeepWalkOL,Grover2016node2vecSF,Tang2015LINELI,高阶邻接嵌入,有向图高阶传递性嵌入}。
符号方法以图论与谱分析为核心,
通过模块度优化\cite{Blondel2008FastUO}、随机游走\cite{随机游走}、谱聚类\cite{谱聚类}、标签传播\cite{标签传播}及图同构检测\cite{图同构检测}等手段揭示图的拓扑规律,具有良好的可解释性,
但在处理大规模、动态与异质图时受限于计算复杂度与人工特征设计。
统计学习方法则通过随机游走\cite{Perozzi2014DeepWalkOL,Grover2016node2vecSF}与矩阵分解\cite{Tang2015LINELI,高阶邻接嵌入,有向图高阶传递性嵌入}等嵌入技术,
将离散图结构映射至连续向量空间,以实现节点、边及子图的低维表征。
然而,这些传统范式普遍存在局部表达受限,在大规模图计算中面临效率与泛化性挑战。

随着图神经网络(GNNs)的发展,图学习领域逐步从传统的机器学习转向深度学习。
然而,过平滑现象(over-smoothing)\cite{Rusch2023ASO}和过压缩问题(over-squashing)\cite{Alon2020OnTB}问题共同限制了 GNNs 在大规模图数据上的表达能力与可扩展性。
为突破这一限制,研究者开始探索以大规模语言模型(如BERT\cite{BERT}, T5\cite{T5}, LLaMA\cite{Llama})为核心的图结构计算新范式。
该方向旨在将图的结构关系与语义信息统一映射至语言空间,使模型具备跨节点、跨关系的全局推理能力。
依托语言模型卓越的语义理解与生成式推理机制,图数据获得了语义补全、自解释与可迁移的潜能,展现出超越传统图神经网络的泛化能力与知识整合能力。
在这一趋势下,图大语言模型(Graph Large Language Models,Graph-LLMs) 逐渐成为连接符号推理与语义建模的关键桥梁。它们试图让语言模型具备理解、推理并生成图结构信息的能力,从而在统一的语义框架中实现图计算与自然语言处理的融合。

尽管 Graph-LLMs 在图语义理解与推理方面取得初步进展,现有研究仍面临一些技术性挑战和局限性,主要集中在以下几个方面:

\textbf{1)图结构编码存在信息缺失与拓扑保真不足:} 现有做法多依赖邻域采样与简化线性化,将图转换为可读序列或片段式子图。此过程易引入起点与遍历路径偏置,边与连通性被截断或重排,环、桥接与长链路等关键结构难以被完整呈现。随之而来的,是对路径可达性、中心性与全局属性的估计偏差,导致在节点分类、链接预测与结构问答等任务上出现系统性性能下滑。

\textbf{2)表示保真与跨模态对齐不足:} 将图转写为文本后,结构信号被弱化为局部片段,缺乏对多阶邻居与全局依赖的统一刻画;同时,语言模型天然缺少图归纳偏置,图与文本嵌入空间难以实现稳定的一一对应。结果表现为:全局与局部语义映射不一致,节点/边级细粒度语义难以锚定到相应词片,跨样本与跨域的表示漂移加剧,模型对复杂拓扑的理解与迁移显著受限。

\textbf{3)推理建模与泛化能力不足:} 以监督微调为主的训练范式往往学习任务捷径而非通用结构规律,模型多给出“结论式”回答而缺少可核验的中间推理链,导致可解释性与稳健性不足。面对未见任务、异构图或尺度变化时,表现出明显的域外退化:对长距离依赖的判断不稳定,对结构扰动与数据漂移敏感,零样本与小样本条件下的迁移能力有限。

尽管存在上述挑战,近年来深度学习技术的演进,尤其是跨模态表征学习与生成式预训练范式的发展,为图大语言模型的进一步突破奠定了基础。
一方面,多模态融合模型在视觉、语音、文本等领域的成功实践表明,大模型具备强大的跨域语义抽象与表示迁移能力,为图模态与语言模态的统一建模提供了新的范式支撑。
另一方面,图结构序列化编码、对比学习与指令微调等方法的引入,使语言模型能够在结构感知的前提下学习图的拓扑依赖与语义关联,提升模型在复杂图任务中的理解与推理能力。

\section{相关研究工作}

对于国内硕士学位论文来说,
一般较少研究完全无前人探索的领域,
所以有必要交待前人在此做出的努力和尝试。
同样,请提供数据和引用保证严谨。

为避免引起评阅老师判定有凑篇幅之嫌,
请有针对的描述前人研究的不足之处,
做到``有破有立''。
\subsection{}

\section{论文研究内容}

此部分必须详细描述,
必要时可划设小节。
国外学位论文的Introduction章基本仅阐述此内容。
为研究开展的相关工作和实验,
此间遇到何难处及对应的解法。
对论文研究领域不甚了解的评阅老师,
希望从摘要和此小节尽可能的了解最多信息。


\section{论文组织结构}

简明扼要的介绍下各章主旨,版面控制半页内。
 % 绪论
\include{contents/whyla}
\include{contents/elem}
\include{contents/sum}  % 总结和展望

% 结尾部分排版
\backmatter

% 引用参考文献数据库
\bibliography{references/test.bib}

% 附录部分
%\appendix
%\chapter{作者简历}

\leftline{教育经历:}

2018.09-2022.06 \quad  \quad 东北师范大学 \quad \quad 软件工程 \quad \quad \quad 本科

2022.09-2025.03 \quad  \quad 浙江大学 \quad \quad \quad \quad 软件工程 \quad \quad \quad 硕士
\\

\leftline{工作经历:}
2022.04-2022.08 \quad  \quad 之江实验室科艺融合研究中心  \quad 实习

2023.09-2025.03 \quad  \quad 浙江大学计算机动画与感知课题组  \quad 实习
\\

\leftline{攻读学位期间发表的论文和完成的工作简历:}

攻读硕士学位期间主要从事计算机视觉、计算机图形学、生成式人工智能相关开发工作,参与了课题组生成式数字人算法的研究和相关系统的开发与设计,累积了比较丰富的开发和科研经验。在此期间发表的相关专利如下:
\begin{enumerate}
    \item 《基于局部关键位置增强的数字人视频生成方法和系统》
    \item 《基于时序位置编码的多模态特征融合数字人视频生成方法和装置》
    \item 《基于细粒度语义描述的手势动作视频生成方法和装置》
\end{enumerate}



% 作者简历
\include{contents/self}

% 致谢
% 致谢不必感谢在下,
% 但请一定感谢清华大学薛瑞尼、
% 机械工程学院陈九历
\include{contents/thanks}

\end{document}
