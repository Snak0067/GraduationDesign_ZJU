\begin{eabstract}
    Currently, with the rapid development of generative artificial intelligence, generative digital humans have gradually widely applied across various industries. It can be utilized in many application scenarios such as short video generation, intelligent customer service systems. Realistic and good interactive generative digital humans can significantly improve the user experience in human-computer interaction.

    Generative digital human modeling relies on multi-dimensional features extracted from raw data, including expressions,hand and full-body gestures, as well as other dimensional features in temporal and spatial dimensions. In current work on generative digital human modeling, most studies have not fully leveraged the advantages of multi-dimensional features, instead using only single-dimensional features extracted from monocular video data sources or fusing only a few dimensional features. This leads to suboptimal digital human modeling results. Additionally, existing work lacks data processing methods that can directly extract multiple dimensional features from monocular videos. Specifically, this has led to current digital human modeling methods facing challenges in four aspects: controllability, generation quality, generation speed, and temporal consistency.
    
    To address the aforementioned problems and challenges, this thesis explores a digital human modeling solution based on multi-dimensional feature fusion. The main contributions include:
    
    1) Fusion and Enhancement of Global Body Features and Local Positional Features:
    Designed and implemented an adversarial generation network architecture based on multi-dimensional feature fusion balances generation quality and speed. For the task of multi-dimensional feature fusion involving global body features and local positional features, a loss function based on a large-scale human body model and a discriminator structure tailored to digital human tasks were developed. These improvements enhance both the local details and the overall appearance in digital human modeling for higher-quality digital human generation.
    
    2) Temporal Consistency Improvement through Fusion of Inter-frame Continuity and Normal Attributes:
    Designed and implemented a temporal feature fusion method based on 3D group convolution and a normal feature fusion method based on multi-task learning. These methods enhance the temporal consistency of generated digital human videos by addressing both frame consistency and cross-frame geometric consistency. Additionally, leveraging the benchmark dataset construction method presented in this thesis, a temporal closed-loop detection method was designed to further evaluate the temporal consistency of the generative digital human videos.
    
    3) Benchmark Dataset Construction Method and Benchmark Digital Human Dataset for Multi-dimensional Feature Fusion:
    Integrated multiple key technologies to achieve high-quality multi-dimensional synthetic digital human data representations extracted from monocular video data. These synthetic image representations accurately depict the digital human’s mouth, eyes, hands and body gestures, facilitating neural network modeling and significantly enhances the controllability of digital humans. Utilizing this benchmark dataset construction scheme, this thesis constructed a character dataset encompassing various scenarios such as speeches, sign language performances, and news broadcasts, covering both half-body and full-body characters. The dataset includes over 30 hours of high-quality digital human video data across 525 digital human identities and corresponding multi-dimensional synthetic images.
    
    Finally, the multi-dimensional feature fusion-based digital human modeling solution developed in this thesis has been adopted by companies such as Hangzhou Cultural Broadcasting Television Group and China Post Consumer Finance Company. For the customer service digital human scenario and the program production digital human scenario, corresponding digital human application solutions were developed, thereby validating the digital human modeling method presented in this thesis.
\end{eabstract}

\ekeywords{Generative Digital Human, Multi-dimensional Feature Fusion, Generative Artificial Intelligence}
    